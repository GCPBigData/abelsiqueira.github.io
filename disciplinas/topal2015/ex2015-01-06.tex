\documentclass{article}
\usepackage[utf8]{inputenc}
\usepackage[T1]{fontenc}
\usepackage[portuguese]{babel}
\usepackage[top=2cm,bottom=2cm,left=1cm,right=1cm]{geometry}
\usepackage{amsmath, amsfonts, amsthm}

\newtheorem{teorema}{Teorema}
\renewcommand{\labelenumi}{(\roman{enumi})}
\newcommand{\K}{\mathbb{K}}

\title{Gabarito do exercício do dia 6 de Janeiro de 2015}
\author{Abel Soares Siqueira}
\date{}

\begin{document}

\maketitle

\begin{teorema}
  Sejam $V$ e $W$ espaços vetoriais e $T \in
  \mathcal{L}(V,W)$ um isomorfismo. Seja $S\subset V$.
  Então,
  \begin{enumerate}
    \item Se $S$ gera $V$, então $T(S)$ gera $W$.
    \item Se $S$ é linearmente independente em $V$,
      então, $T(S)$ é linearmente independente em $W$.
    \item Se $S$ é base de $V$, então $T(S)$ é base de
      $W$.
  \end{enumerate}
\end{teorema}
\begin{proof}
  Note que $T$ isomorfismo quer dizer que $T$ é injetora e sobrejetora.
  Note também que $S$ não necessariamente é finito, nem que $V$ necessariamente
  tem dimensão finita.
  \begin{enumerate}
    \item
    \begin{description}
      \item[$(\Rightarrow)$]
        Tome $w\in W$.
        Como $T$ é sobrejetora, existe $v\in V$ tal que $Tv=w$.
        Daí, como $S$ gera $V$, existem $v_1,\dots,v_n\in S$ e
        $\alpha_1,\dots,\alpha_n\in\K$ tal que
        $$ v = \alpha_1v_1 + \dots + \alpha_nv_n. $$
        Daí
        \begin{align*}
          w & = Tv \\
            & = T(\alpha_1v_1 + \dots + \alpha_nv_n) \\
            & = \alpha_1Tv_1 + \dots + \alpha_nTv_n.
        \end{align*}
        Como $Tv_j\in T(S), j = 1,\dots,n$, então $w\in T(S)$.
        Como $w\in W$ foi arbitrário, temos $W\subset T(S)$.

      \item[$(\Leftarrow)$]
        Tome $v\in V$. Seja $w=Tv$. Como
        $Tv\in W=[T(S)]$, existem $w_1,\dots,w_n\in T(S)$ e
        $\alpha_1,\dots,\alpha_n\in\K$ tais que
        \begin{align*}
          Tv & = \alpha_1w_1+\dots+\alpha_nw_n.
        \end{align*}
        Como $w_j\in T(S)$, existem $v_j\in S$ tais que
        $Tv_j = w_j$, daí
        \begin{align*}
          Tv & = \alpha_1Tv_1+\dots+\alpha_nTv_n \\
          & = T(\alpha_1v_1+\dots+\alpha_nv_n).
        \end{align*}
        Como $T$ é injetora, temos
        $$ v = \alpha_1v_1 + \dots + \alpha_nv_n, $$
        isto é, $v\in [S]$. Como $v\in V$ foi arbitrário, temos
        $V\subset[S]$.
    \end{description}
    \item
    \begin{description}
      \item[$(\Rightarrow)$]
        Tome $w_1,\dots,w_n \in T(S)$ distintos, e sejam
        $\alpha_1,\dots,\alpha_n\in\K$ tais que
        $$ \alpha_1w_1 + \dots + \alpha_nw_n = 0.$$
        Para cada $w_j$ existe $v_j\in S$ tal que $w_j = Tv_j$.
        $$ \alpha_1Tv_1 + \dots + \alpha_nTv_n = 0.$$
        Logo,
        $$ T(\alpha_1v_1 + \dots + \alpha_nv_n) = 0.$$
        Como $T$ é injetora e $T(0) = 0$, temos
        $$ \alpha_1v_1 + \dots + \alpha_nv_n = 0. $$
        Como $T$ é injetora, e $w_j\neq w_i, j\neq i$, então
        $v_j\neq v_i, j\neq i$. Logo, $v_1,\dots,v_n$ são distintos.
        Como $v_j\in S$, e por hipótese, $S$ é linearmente independente,
        então toda combinação linear nula de elemtos de $S$ deve
        ter os coeficientes nulos, ou seja,
        $\alpha_1 = \cdots = \alpha_n = 0$.
        Portanto, os vetores $w_1,\dots,w_n$ são linearmente independentes.
        Como a escolha desses vetores foi arbitrária, toda escolha finita de
        vetores de $T(S)$ é linearmente independente, de modo que $T(S)$ é
        linearmente independente.

      \item[$(\Leftarrow)$]
        Tome $v_1,\dots,v_n \in S$ distintos, e sejam
        $\alpha_1,\dots,\alpha_n\in\K$ tais que
        $$ \alpha_1v_1 + \dots + \alpha_nv_n = 0.$$
        Daí,
        $$ T(\alpha_1v_1 + \dots + \alpha_nv_n) = T(0) = 0,$$
        isto é,
        $$ \alpha_1Tv_1 + \dots + \alpha_nTv_n = 0,$$
        e como $T$ é injetora, $Tv_1,\dots,Tv_n$ são distintos.
        Mas $Tv_j\in T(S), j=1,\dots,n$, e por hipótese, $T(S)$ é linearmente
        independente,
        de modo que toda combinação linear nula deve
        ter os coeficientes nulos, ou seja,
        $\alpha_1 = \cdots = \alpha_n = 0$.
        Portanto, os vetores $v_1,\dots,v_n$ são linearmente independentes.
        Como a escolha desses vetores foi arbitrária, toda escolha finita de
        vetores de $S$ é linearmente independente, de modo que $S$ é
        linearmente independente.
    \end{description}
    \item Um conjunto é base se gera o espaço e é linearmente independente,
      então o resultado segue diretamente de (i) e (ii).
  \end{enumerate}
\end{proof}

\end{document}
