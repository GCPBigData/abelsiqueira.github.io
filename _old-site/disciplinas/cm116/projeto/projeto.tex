\documentclass[a4paper,12pt]{article}
\usepackage[utf8]{inputenc}
\usepackage[T1]{fontenc}
\usepackage{hyperref}
\usepackage[portuguese]{babel}
\usepackage{algorithm,algorithmic}
\usepackage{amsmath,amsfonts,amsthm}

% Esses padrões não se adequam na ABNT
\usepackage[left=2cm,right=2cm,top=3cm,bottom=2cm]{geometry}
\usepackage[style=ieee,backend=bibtex,sorting=none]{biblatex}
\DefineBibliographyStrings{portuguese}{
  references = {Bibliografia},
}
\addbibresource{refs.bib}

% Um pacote de comandos auxiliares
\usepackage{auxiliares}

\title{Modelo e explicação do projeto}
\author{Abel Soares Siqueira}
\date{Dia do mês do ano}

\begin{document}

\maketitle

\begin{abstract}
  Aqui vai um pequeno resumo do trabalho.
  O resumo deve ter no máximo 15 linhas, sendo que menos é melhor.

  O trabalho inteiro deve ter ao menos 5 páginas e no máximo 10.
\end{abstract}

\section{Introdução}

Uma pequena introdução. O que significa o modelo; de onde vem; como chega nele;
etc. O que for pertinente e factível.

\section{Discretizações/Aplicação dos métodos}

Explique as discretizações, ou as aplicações dos métodos, como ficaram no
problema específico, como foram resolvidos, etc.

\subsection{Cronograma dos projetos}

\begin{itemize}
  \item Primeira parte (13 de Abril)
    \begin{itemize}
      \item Modelo PVI;
      \item Métodos Euler (Explícito e implícito), Runge-Kuttas, Outros;
      \item Desenhos;
    \end{itemize}
  \item Segunda parte (11 de Maio)
    \begin{itemize}
      \item Modelo PVC;
      \item Diferenças finitas de ordem 1 e 2;
      \item Desenhos;
      \item Notebook;
    \end{itemize}
  \item Terceira parte (29 de Junho)
    \begin{itemize}
      \item Modelo para EDP;
      \item Várias opções. Crank-Nicholson;
      \item Desenhos;
      \item Animações.
    \end{itemize}
\end{itemize}

Exemplos de PVI:
\begin{itemize}
  \item Equações Presa-Predador;
  \item Pendûlos;
\item Aceleração com atrito do ar.
\end{itemize}

Exemplos de PVC:
\begin{itemize}
  \item Equação do calor escionária com geração interna de calor;
  \item Flexão de uma Viga;
  \item Equação de Laplace bidimensional.
\end{itemize}

Exemplos de EDP:
\begin{itemize}
  \item Equação do calor em duas dimensões com geração interna;
  \item Equações de águas rasas;
  \item Equação de Black-Scholes;
  \item Alguma aplicação da equação de Fisher-Kolmogorov;
\end{itemize}

\section{Explicando LaTeX}

Existem dois tipos de código matemático: inline ou display.
O inline é junto com o texto, como $f(x) = x^2$, e o display aparece
centralizada separado do texto, como $$f(x) = x^2.$$
É importante notar que as regras de português continuam valendo mesmo com código
matemático em display (veja a pontuação e parágrafo).

Para o display, existem três formas de colocar o código. O
\begin{verbatim}
$$ f(x) = x^2 $$
\end{verbatim}
o
\begin{verbatim}
\begin{equation}
  codigo
\end{equation}
\end{verbatim}
e o
\begin{verbatim}
\begin{align}
  codigo
\end{align}
\end{verbatim}
O primeiro e o segundo fazem a mesma coisa, sendo que o primeiro é melhor para
códigos curtos. O terceiro deixa você colocar códigos multi-linhas, e o segundo
e o terceiro colocam numeração para as equações.
Você pode remover essa numeração colocando um * depois de \verb+equation+ ou
de \verb+align+.
Numere apenas as equações que for referenciar. Veja exemplos no código.

\subsection{Exemplos de código}
\begin{equation}
  ax^2 + bx + c = 0, \qquad a \neq 0 \label{eq:quadratica}
\end{equation}
A Equação \eqref{eq:quadratica} é a equação de Bháskara.
Para resolver, completamos quadrados
\begin{align}
  ax^2 + bx + c & = a\bigg(x^2 + \frac{b}{a}x\bigg) + c \nonumber \\
                & = a\bigg[\bigg(x + \frac{b}{2a}\bigg)^2 -
\frac{b^2}{4a^2}\bigg] + c = 0. \label{eq:comp.quad}
\end{align}
Isolando a parte quadrática na Equação \eqref{eq:comp.quad} temos
\begin{equation*}
  \bigg(x + \frac{b}{2a}\bigg)^2 = \frac{b^2}{4a^2} - \frac{c}{a},
\end{equation*}
que pode ser escrito como
\begin{equation}
  \bigg(x + \frac{b}{2a}\bigg)^2 = \frac{b^2 - 4ac}{4a^2}.
  \label{eq:isolado}
\end{equation}
Se $b^2 - 4ac > 0$ na Equação \eqref{eq:isolado}, essa equação tem duas
soluções, que podemos escrever como
\begin{align*}
  x & = -\frac{b}{2a} \pm \sqrt{\frac{b^2 - 4ac}{4a^2}} \\
    & = -\frac{b}{2a} \pm \frac{\sqrt{b^2 - 4ac}}{2\modulo{a}} \\
    & = -\frac{b}{2a} \pm \frac{\sqrt{b^2 - 4ac}}{2a},
\end{align*}
onde o módulo some pois o $\pm$ já engloba as possibilidades do módulo.
Isso se resumo a
\begin{equation}
  x  = \frac{-b \pm \sqrt{b^2 - 4ac}}{2a}.
  \label{eq:bhaskara}
\end{equation}
A Equação \eqref{eq:bhaskara} é dita fórmula de Bháskara.

\section{Resultados Computacionais (O nome pode ser diferente)}

Após implementarem os métodos, compare com a solução exata, quando houver.
Faça os gráficos que forem necessários, e pode fazer animações quando for
possível.

\section{Conclusão}

Depois de tudo você conclui.

% Para citas sem referenciar no texto, use
\nocite{otimizacao:nocedal}
\printbibliography

\end{document}
